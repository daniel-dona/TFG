\section{Glosario}

\begin{itemize}
    
\item IA: Acrónimo de Inteligencia artificial, disciplina que intenta imitar o emular comportamientos o procesos que se consideran inteligentes. 

\item Aprendizaje Computacional: rama que estudia mecanismos que permitan a un sistema mejorar a partir de la experiencia. Habitualmente se suele relacionar con la resolución de problemas planteados en el ámbito de la Inteligencia Artificial.

\item TTS: Acrónimo del inglés "text to speech", método de síntesis de voz mediante una entrada de texto.

\item Deepfake: Acrónimo del inglés "deep learning fake", falsificación generada total o parcialmente mediante el uso de modelos de inteligencia artificial.

\item Entrenar (modelos): Proceso por el cual un modelo de Aprendizaje Computacional extrae características o información relevante de un conjunto de datos para adquirir alguna capacidad funcional relacionada con dichos datos.

\item Fonema: Unidad sonora mínima articulable en una palabra, el proceso del habla humana.

\item Toolkit: Del inglés "conjunto de herramientas", habitualmente una recolección de distintos componentes de software o herramientas que se usan de forma conjunta o que su uso pertenece a un mismo dominio, facilitando así el cumplimiento de ciertas tareas en un mismo ecosistema.

\item Notebook (Google Colab, Jupyter, etc...): Tipo de fichero especial que contiene una mezcla de código y explicaciones, habitualmente empleados en contextos de docencia y en demostraciones guiadas para exponer el funcionamiento de una pieza de software.

\item Código abierto: forma de licencia para la publicación del código fuente de un producto de software que se caracteriza por dar especial libertad al usuario para su examen, modificación o uso en nuevas creaciones.

\item Dataset: Conjunto de datos especialmente pensado y preparado para su procesamiento de forma automatizada y para fines diversos. 



\end{itemize}
\newpage 

\listoffigures

\newpage 