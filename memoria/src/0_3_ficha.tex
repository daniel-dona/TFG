\thispagestyle{empty}

{\scshape\large Ficha del trabajo \par}
\vspace{0.5cm}

\def\arraystretch{2}
\begin{center}
\begin{tabularx}{1\textwidth} { 
  | >{\raggedleft\arraybackslash}X
  | >{\raggedright\arraybackslash}X | }
\hline
Título del trabajo: & Generación y detección de falsificaciones de voz en idioma castellano. \\
\hline
Nombre del autor: & Daniel Doña Álvarez  \\
\hline
Nombre del consultor: & Ferran Diego Andilla \\
\hline
Nombre del PRA: & Carles Ventura Royo \\
\hline
Fecha de entrega (mm/aaaa): & 06/2022 \\
\hline
Titulación: & Grado en Ingeniería Informática \\
\hline
Área del Trabajo Final: & Área de Inteligencia Artificial \\
\hline
Idioma del trabajo: & Castellano \\
\hline
Número de créditos: & 12 \\
\hline
Palabras clave: & deepfake, tts, voice synthesis, deepfake detection \\
\hline

\end{tabularx}
\end{center}

\newpage
\thispagestyle{empty}

{\scshape\large Ficha del trabajo \par}
\vspace{0.5cm}
\def\arraystretch{2}
\begin{center}
\begin{tabularx}{1\textwidth} { 
  | >{\raggedleft\arraybackslash}X
  | >{\raggedright\arraybackslash}X | }
\hline
\multicolumn{2}{|c|}{Resumen}\\
\hline
\multicolumn{2}{|X|}
{

El problema de la confianza en la fiabilidad de la información no es un problema nuevo, no es ni un problema realmente asociado a la omnipresencia contemporánea de la información. Siempre que se ha podido afirmar que una información es veraz es porque existía la posibilidad de que también fuese falsa. 
\newline\newline
Ante la duda sobre qué es cierto y qué es falso tendemos a creer en la voz de aquellos a los que atribuimos cierta autoridad o cierta confianza personal. Pero esta forma de verificación de la información cada día está en una situación de mayor peligro.
\newline\newline
Hace décadas que conseguimos hacer que los computadores sintetizasen voz, pero no habríamos imaginado que un computador -o alguien que lo use  con mala fe- podría imitar lo suficientemente bien la voz de una persona particular como para suplantarla.
\newline\newline
Nuestro trabaja examina el estado del arte en la síntesis de voz, las posibilidades reales de que cualquiera sea suplantado y establece los siguientes pasos a dar para la futura detección de estas falsificaciones que se conocen ya popularmente como “deep-fakes”.

} \\
\hline
\end{tabularx}
\end{center}

\begin{center}
\begin{tabularx}{1\textwidth} { 
  | >{\raggedleft\arraybackslash}X
  | >{\raggedright\arraybackslash}X | }
\hline

\multicolumn{2}{|c|}{Abstract}\\
\hline

\multicolumn{2}{|X|}
{
The problem of confidence in the reliability of information is not a new problem, nor is it a problem really associated with the contemporary omnipresence of information. Whenever it has been possible to affirm that information is true, it is because there was always the possibility that it could also be false.
\newline\newline
When in doubt about what is true and what is false, we tend to believe the voice of those to whom we attribute a certain authority or a certain personal trust. But this form of information verification is becoming more and more endangered every day.
\newline\newline
We were able to make computers synthesize voice decades ago, but we would not have imagined that a computer - or someone using it in bad faith - could mimic a particular person's voice well enough to impersonate it.
\newline\newline
Our research examines the state of the art in voice synthesis, the real possibilities of anyone being impersonated, and sets out the next steps for future detection of these fakes, which are now popularly known as "deep-fakes".
} \\

\hline

\end{tabularx}
\end{center}


\restoregeometry
\newpage